	% Compile with XeLaTeX, TeXLive 2013 or more recent
\documentclass{beamer}

% Base packages
\usepackage{fontspec}
\usepackage{xunicode}
\usepackage{xltxtra}

\usepackage{amsfonts}
\usepackage{amsmath}
\usepackage{longtable}
\usepackage{csquotes}
\usepackage{standalone}

\usepackage{listings}
\lstset{basicstyle=\footnotesize\ttfamily, breaklines=true, keepspaces=true }

% Setup Russian hyphenation
\usepackage{polyglossia}
\setdefaultlanguage[spelling=modern]{russian} % for polyglossia
\setotherlanguage{english} % for polyglossia
\defaultfontfeatures{Scale=MatchLowercase, Mapping=tex-text}

% Setup fonts
\newfontfamily\russianfont{CMU Serif}
\setromanfont{CMU Serif}
\setsansfont{CMU Sans Serif}
\setmonofont{CMU Typewriter Text}

% Be able to insert hyperlinks
\usepackage{hyperref}
\hypersetup{colorlinks=true, linkcolor=black, filecolor=black, citecolor=black, urlcolor=blue , pdfauthor=Evgeny Yulyugin <yulyugin@gmail.com>, pdftitle=Параллельное программирование}
% \usepackage{url}

% Misc optional packages
\usepackage{underscore}
\usepackage{amsthm}

% A new command to mark not done places
\newcommand{\todo}[1][Write me]{{\color{red}TODO\ #1}}

\newcommand{\abbr}{\textit{англ.}\ }

\title{Основы MPI}
\subtitle{Курс «Параллельное программирование»}
\subject{Lecture}
\author[Евгений Юлюгин]{Евгений Юлюгин \\ \small{\href{mailto:yulyugin@gmail.com}{yulyugin@gmail.com}}}
\date{\today}
\pgfdeclareimage[height=0.5cm]{mipt-logo}{../common-images/mipt.png}
\logo{\pgfuseimage{mipt-logo}}

\typeout{Copyright 2014 Evgeny Yulyugin}

\usetheme{Berlin}
\setbeamertemplate{navigation symbols}{}%remove navigation symbols

\begin{document}

\begin{frame}
\titlepage
\end{frame}

\section{Обзор}

\begin{frame}
\tableofcontents
\end{frame} 

\begin{frame}{История MPI}

\begin{itemize}
	\item MPI расшифровывается как интерфейс передачи сообщений (\abbr Massage Passing Interface).
	\item Программный интерфейс, позволяющий обмениваться сообщениями между процессами выполняющими одну задачу.
	\item Разработан Уильямом Гроуппом, Эвином Ласком и другими.
	\item Первая версия разрабатывалась в 1993-1994 году.
	\item MPI версии 1 вышла в 1994 году.
	\item MPI 1.1 опубликован в 12 июня 1995 года. Поддерживается большинством современных реализаций MPI. Первая раелизация появилась в 2002 году.
	\item Существуют реализации для языков Fortran, Java, C и C++.
\end{itemize}

\end{frame}

\section{Базовые функции MPI}

\defverbatim[colored]\initfinalize{
\begin{lstlisting}[language=C++,basicstyle=\ttfamily,keywordstyle=\color{blue}]
int MPI_Init(int *pargc, char ***pargv);
int MPI_Finalize(void);
\end{lstlisting}
}

\begin{frame}{Инициализация и завершение процессов}

Определены в заголовочном файле mpi.h

\vfill

\initfinalize

\end{frame}

\defverbatim[colored]\size{
\begin{lstlisting}[language=C++,basicstyle=\ttfamily,keywordstyle=\color{blue}]
MPI_Comm_size(MPI_Comm comm, int *size);
\end{lstlisting}
}

\defverbatim[colored]\rank{
\begin{lstlisting}[language=C++,basicstyle=\ttfamily,keywordstyle=\color{blue}]
MPI_Comm_rank(MPI_Comm comm, int *rank);
\end{lstlisting}
}

\begin{frame}{MPI communicator}

\begin{itemize}
	\item MPI_COMM_WORLD is the initially defined universe intracommunicator for all processes to conduct various communications once MPI_INIT has been called,
	\item MPI_COMM_SELF,
	\item User defined communicators.
\end{itemize}

\end{frame}

\begin{frame}{Self-identification}

\size

Returns the size of the group associated with communicator.

\vfill

\rank

Determines the rank of the calling process in the communicator.

\end{frame}

\section{Конец}
% The final "thank you" frame 
\begin{frame}

{\huge{Спасибо за внимание!}\par}

\vfill

\tiny{\textit{Замечание}: все торговые марки и логотипы, использованные в данном материале, являются собственностью их владельцев. Представленная здесь точка зрения отражает личное мнение автора, не выступающего от лица какой-либо организации.}

\end{frame}

\end{document}